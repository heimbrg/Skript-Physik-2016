\documentclass[11pt]{scrbook}

\usepackage[utf8]{inputenc}
\usepackage[ngerman]{babel}
\usepackage{skript2016}



\begin{document}

%leere Seite
\newpage
\thispagestyle{empty}
\mbox{}

\chapter*{WIE SIE MIT DIESEM SKRIPT ARBEITEN}
\section{Lernen im Kontext}
Jedes Kapitel beginnt mit der Betrachtung einer besonderen Situation in welcher die Physik gebraucht wird oder studiert werden kann. Danach wird die Physik, die Sie benötigen um diesen Kontext zu untersuchen nach und nach beschrieben. \newline

Die Situationen wurde so ausgewählt, dass Ihnen diese zeigen sollen, wie die Physik das Leben der Menschen verbessern kann, wo die Physik in Ingenieurwesen und Technologie verwendet wird und wie die Physik unser Verständnis der Welt auf fundamentaler Ebene erweitert.

In jedem Kapitel werden Sie ihr Kenntnisse und Fertigkeiten in einem oder mehreren Gebieten der Physik erweitern und vertiefen. In späteren Kapiteln werden Sie viele der grundlegenden Ideen dieser Gebiete - in völlig anderen Kontexten - wieder antreffen und vertiefen. So werden Sie Ihr Wissen Schritt für Schritt aufbauen und lernen, die Grundlegenden Prinzipien der Physik in unterschiedlichen Kontexten anzuwenden. In jedem Kapitel finden Sie die folgenden Dinge:

\section{Haupttext}
Er präsentiert den Kontext jeder Unterrichtseinheit und erklärt die relevante Physik dort wo Sie diese benötigen. \newline

Innerhalb des Haupttextes sind einige Wörter \textbf{fett} gedruckt. Dies sind die Schlüssel\-begriffe der Physik. Ich empfehle Ihnen eine Zusammenfassung/Erklärung dieser Begriffe (und anderer wenn erforderlich) zu erstellen, wenn Sie diese antreffen. So können Sie in späteren Kapiteln oder in der Vorbereitung auf Prüfungen darauf zurückgreifen. 

\section{Aufgaben}
Der Text verweist auf viele Aufgaben. Diese beinhalten praktische Arbeiten, die Benutzung von ICT, Lesen, Schreiben, den Umgang mit Daten und Diskussionen. Einige Aufgaben machen Sie am Besten zu zweit oder in einer kleinen Gruppe, andere sollten Sie selbstständig erledigen. Zu einigen Aufgaben erhalten Sie zusätzliche Informationen.

\begin{Activity}[Ungleichförmige Bewegung]%Anpassen! U.a. Nummer der Aufgabe, Farbe, Balken, evtl Symbol
Benutzen Sie ein Aufnahmegerät, um Ihre Bewegung aus der Hocke in einen Sprint zu filmen. Erstellen Sie anhand des Videos ein Diagramm (Ort - Zeit).

Berechnen Sie Ihre Geschwindigkeit und Ihre Beschleunigung in jedem kurzen Zeitintervall zwischen jeweils 2 Frames.
\end{Activity}

\section{Fragen}
Sie finden zwei Arten von Fragen in diesem Skript:
\begin{itemize}
\item Einige Fragen sind gedacht um sie während der Bearbeitung des Skripts, zur Selbstkontrolle oder als Hausaufgabe zu beantworten. Die Lösungen finden Sie jeweils am Ende des Kapitels.
\item Fragen zum ganzen Kapitel dienen der Zusammenfassung des Gelernten, die Lösungen dazu erhalten Sie von mir.
\end{itemize}
Haben Sie eine Frage beantwortet, überprüfen Sie die Antwort. Ist Ihre Antwort nicht korrekt, so benutzen Sie die Lösung (und den entsprechenden relevanten Abschnitt im Skript) um Ihre Antwort zu korrigieren. 

\section{Verweise auf mathematische Hintergründe}
Diese Textbox am Seitenrand verweist Sie jeweils auf das entsprechende mathematische Hinter\-grund\-wissen. im Anhang des Skripts. Dieses Hinter\-grund\-wissen wird für die Physik benötigt. Es kann das Umformen von Gleichungen, bestimmte Berechnungen, das Zeichnen von Graphen, usw. enthalten. Die meisten dieser Themen sollten aus dem Mathematikunterricht bekannt sein. 

\section{Verweise auf praktische Fertigkeiten}
Diese Textbox am Seitenrand verweist Sie auf Hinweise zu praktischen Fertigkeiten am Ende des Skripts. Diese Hinweise dienen als Anweisung zum praktischen Arbeiten. 

\section{Vertiefungen}
Diese Textbox am Seitenrand verweist auf vertiefende Texte oder weist auf Verknüpfungen innerhalb des Skripts hin.

\section{Zusammenfassung}
Am Ende jedes Kapitels werden die wichtigsten Begriffe und Grössen wiederholt und zusammengefasst.

\chapter*{Wissenschaftliches Arbeiten}
Im Verlaufe Ihrer Ausbildung im Grundlagenfach Physik werden Sie ein Verständnis des wissenschaftlichen Arbeitens gewinnen. Sie werden gewisse Fertigkeiten erlernen, die es braucht um im Bereich der Physik wissenschaftlich arbeiten zu können. Die Beispiele und Aufgaben im Skript und die Übungen die im Unterricht ausgeteilt werden, helfen Ihnen eine Sicherheit im 

Throughout this course, you will be developing your
knowledge and understanding of what it means to work
scientifically in the context of physics and its applications.
The examples, questions and activities in each chapter will
help you to become increasingly competent in manipulating
quantities and their units, in planning, carrying out and
evaluating practical experiments, and in communicating your
knowledge and understanding of physics. The Maths notes
are written to help you to build up your skills in handling
quantities and units, and the Practical skills references are
designed to support your investigative and practical work.
You will also learn more about the ways in which the scientific
community, and society as a whole, use physics and contribute
to its progress. You will study many applications of physics in
a wide variety of situations, and you will see how the risks and
benefits of those applications are evaluated. You will also learn
more about the ways in which the scientific community operates
to validate new knowledge relating to physics, and to ensure that
work on physics and its applications is carried out with integrity.


\end{document}
